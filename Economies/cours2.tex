\paragraph{Forme de contrainte} : Politiques, Economique, socio-culturels, technologique, écologiques et légal. Le PESTEL sert donc à montrer les contraintes d'un secteurs dans tout ces domaines.
Ces forces sont tous interdépendantes : l'une infecte l'autre. Si on modifie l'aspect Technologique, on modifie le goût des utilisateurs ainsi que les aspects économiques du marché. 
On a aussi des frontières difficiles à cerner, par exemple entre le Politique et le Légale.
\begin{description}
	\item[Le Légale] Droit de la concurrence qui réglemente le jeu concurentiel. Elle gère les monopoles. (abue de position dominante), ainsi que les ententes : les ententes sur les prix, et les ententes entre les concurrents, ainsi que les ententes sur le lieu géographique. Un grand nombre d'ententes ne concernent pas les utilisateurs directe mais plus les marchés des entreprises. La difficulté de l'entente est de prouver qu'il y a bien eu une entente (il pourrait y avoir du mimétisme qui ne sont pas des ententes). La meilleur chose que l'on peut faire est de donner de la suspition dans l'entente, par exemple de donner l'immunité à celui qui dénonce les autres. Le droit du travail : un certain nombre de droit nous protège (CDI, CDD, travaille des enfants, etc.). Ilgère aussi les devoirs des salariés. 
Législation sanitaire concerne les traitements et les matériaux sélectionner pour un produit (exemple de la viande de cheval). Les pays choisissent surtout aussi les normes pour favoriser les entreprises de ce pays (forme de protectionnisme déguisé).
	\item[Politique] Les décisions politiques ce manifeste souvent dans un acte légal, il est donc difficile de les différenciers. Cependant, on souhaite une stabilité gouvernementale (un gouvernement stable), une faible fiscalité (les taux d'impositions les plus bas, 34\% des bénéfices des entreprises vont à l'état, sans oublié les nombreuses taxes (sur la valeurs ajouté par exemple)). Mais il y a quand même une optimisation fiscal : supposons la Suisse, la France et Loréal. Si loréal dépose ses marques, ses brevets en Suisse et produit en France, il renverra 90\% des bénéfice en Suisse pour l'utilisation des brevets, il reste donc très peu pour la fiscalité. Il y a cependant une concurrence des politiques fiscales entre les pays. Par exemple le crédit d'impot recherche permet de donner des crédits aux entreprises pour la recherche. Il est systématiquement sous estimé dans les prévisions des dépenses, l'état se fait donc à chaque voir avoir. Si on subit des pertes durant les années passé, la fiscalité fiscale déduis des impots les pertes.
Le protectionnisme libéralisme est de savoir si on autorise les importations / exportations facilement.
Protection sociale : Quand il y a une protection sociale (par exemple la sécurité sociale contre les risques maladies, de la vieillesse et de la grosesses). Il y a aussi le risque du chomâge (passage de ASSEDIC à Pole emploie). Il y a les complémentaires privé et associative qui viennent renforcé ses protections sociales. Elles sont contribué par les entreprises, les cotisations salariales, etc... Aux USA, tout ses problèmes doivent être géré uniquement par l'individus.
Corruption : Donnes-t-on des pots de vins ? Sur les marchés publiques, la France n'est pas très bien reconnu (beaucoup de corruptions). 
	\item[Economique] Le contexte macro économqiue est-elle favorable à l'entreprises ? Le cycle économique du PIB est donc importants. 
Les taux d'intérêt en France est assez faible et favorable aux entreprises. La banque central Européenne et Amériques ont décidé de rendre l'argent faibles. Quand la banque central décide de baisser ses taux d'intérêt, cela va avoir un impact sur les entreprises et les ménages. Une politique monétaire favorable permet de préter plus facilement aux entreprises. Elles peuvent cependant choisir à qui elles prêtent, etc. Les banques commerciales et européennes ont subit tellement de problème en 2008, 2009 qu'aujourd'hui elles prêtent beaucoup moins.
L'inflation en France est assez faibles (ce qui augmente en ce moment est uniquement le carburant et le tabac). Le risque n'est pas vraiment l'inflation mais l'hyper inflation, où les prix augmentais en flêche. À l'inverse, il y a aussi la déflation qui est un problème : les marges de l'entreprise baisse (et doit donc licencier). L'autre problème est que la baisse des prix entrennent la baisse des prix : les gens attendent que les prix baisses encore plus.
Le chômage : l'existence d'un grand nombre de chômage permet par exemple de mieux négocier les salaires pour les salariés. Les revenus disponibles est ce qui reste aux salariés.
	\item[Socioculturel] Y a t il que des jeunes, ou la population est-elle vieillissante ? SI on a une population très jeunes, il y a des secteurs d'activité qui prédomine. En france, le marché de la dépendance fonctionne très bien (maison de retraites par exemple). Il faut donc adapté les activité à la démographie du pays.
La distributions des revenus : y-a-t-il une classe moyenne, y a-t-il que des pauvres très pauvre et des riches très riche etc...
La mobilité sociale / changements de mode de vie (par exemple modifier par la technologie).
Arbitrage de loisir / travail (dû aux vacances).
Le Consumérisme : l'exigence des consummateurs : plus vous êtes éduqué, et plus vous êtes éxigeant ! 
	\item[Technologique] La dépenses publiques et privé de la recherche et du dévelopment. 
		Nouvelles découvertes.
		Vitesse des transferts technologiques (capacité d'absorption)
		Taux d'obsolescence.
		Protection de la propriété.
	\item[Ecologique] Lois sur la protection de l'environnement
		Retraitement des déchets
		Consommation d'énergies.
\end{description}

\paragraph{Les tendances structurelles} Identifier les contraintes stratégiquement les plus pertinentes à suivre pour le futur du point de vue d'une entreprise du marché. Le développement est la somme de la globalisation, des développement technoscientifique et du libéralisations des marchés. Il faut aussi avoir une contrainte de rentabilité, sociale (égalité homme / femme, pas de travaille des enfants, temps de travail normal, reconnaissance des syndicats) et le respect écologique.
\paragraph{Elaborer des scénarios} C'est la représentation plausible du futur (mais n'est pas unique). 
\begin{itemize}
	\item Evolution de l'environnement incertaine
	\item Scénario représentation plausible et détaillée de différents futurs envisageables, obtenue à partir de la combinaison de tendances structurelles très incertaines.
	\item Trois étapes \begin{itemize}
			\item Identifier les quelques forces qui ont l'impact potentiel le plus élevé et dont l'évolution est fortement incertaines
			\item Combiner chaque possible pour élaborer des scénarios, du type 1 pessimiste, 1 optimiste, 1 moyen
			\item Déterminer la probabilité d'occurrence pour chaque scénario
		\end{itemize}
\end{itemize}
