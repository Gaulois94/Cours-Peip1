\chapter{Introduction}
\paragraph{} Pendant très longtemps, les penseurs économiques, datant du $18^e$ siècles considéraient que les activités créatiques de valeurs s'intéressaient à l'agriculture. Aujourd'hui, on a une vision plus large de la création de richesses : il n'y a pas que la nourriture qui créer de la richesse : il y a aussi les industries et la création de services.
\paragraph{} Richesse privé : entreprises dégagent une marge. Sur qui prélève-t-on les richesses pour les services publiques.

\section{L'entreprise, un lieu de production}

\paragraph{} Une entreprise combine des facteurs de ressources : travail, capital (avec emprunt), consommation intermédiaires comme les fournitures, les énergies, etc. Ils consomment aussi de l'informations, surtout sur le comportement de leurs concurrents : ils scannent en permanence leurs secteurs, essayent d'avoir de l'informations. On ne peut esquiver les informations : les informations internes (des employés par exemple) tout comme les informations extérieurs sont très importantes. Les  entreprises de services ont plus besoin de travail que de capital, car on peu besoin de matériels couteux.
\paragraph{} Le but de l'entreprises est de produire des biens pour les industries, et les services pour les entreprises de services. L'objectifs de l'entreprise est de créer de la valeurs. En premier, la valeurs ajoutés qui est un concept très important pour les entreprises ou des sociétés. C'est ce qu'il reste une fois avoir payés les ressources. Le PIB est la sommes de toute les valeurs ajoutés de toutes les organisations. La croissance économique est tout simplement la croissance des valeurs ajoutés. Chaque année on mesure la production de richesse créer : la valeurs ajoutés est donc un flux : elle baisse ou augmente selon les années / prospérités. Les entreprises créer de la valeurs d'abord pour elle même : C'est ce qu'il reste pour augmenter ses richesses. Elle peut se permettre d'être à la baisse mais cela ne peut durer. On peut être plus cher que la concurrence uniquement si la qualités de la prestation est de meilleurs qualités. Un prix faible est souvent considéré comme quelque chose de bas de gamme comparé à la concurrence (ce qui n'est pas toujours le cas).
\paragraph{} L'utilisateurs / usagés de services publiques doivent en sortir valorisés. Par exemple, si on suit des cours durant 5 ans, c'est pour gagner / valorisé notre salaires, comparé à un travail immédiat. Mécanisme de cadeaux est totalement absurde car détruit de la valeurs : prix estimé par celui qui reçoit $<$ prix réel.

\section{Son objectifs est de créer de la valeurs}

\paragraph{} La pertinence pour les étudiants c'est pouvoir se consacré à 90 \% de ses études sur 5 ans, de se donner les moyens d'atteindre ses objectifs. Les entreprises doivent donc se donner des moyens (financiers, humains, stratégiques (conquérire des clients, rachetés des entreprises...)) pour atteindre ses objectifs
\paragraph{} L'efficience c'est faire mieux avec moins : avec moins de moyens on fait plus de valeurs, plus de création de richesses. Les moyens que l'on dispose sont : 
\begin{itemize}
	\item : productivité du travail des salariés.
	\item : on fait plus avec moins de salariés.
	\item : on mutualise : on des pièces créer sur la même plateforme par exemple, comme chez renault avec ses moteurs.
\end{itemize}
\paragraph{} L'efficacité c'est es-ce qu'on à atteint nos objectifs voir les dépassées.

\section{La répartition de la valeurs}

\paragraph{} L'entreprises doit répartir les richesses pour les salaires, l'intéressement (affinité des salariés), les primes, etc. L'entreprise doit aussi repartir les richesses avec les rapporteurs de capitaux : ceux qui sont propriétaires des entreprises et de son capital : les actionnaires, et ceux qui prêtent le capital : ceux qui prêtent des fonds à court terme par exemple sont des préteurs. Pour les prêt sur le long terme, dans le cas des petites entreprises des banques, ou les marchés financiers pour les grosses entreprises en générales : se sont les créanciers. Il prêtent selon des durées données, des contrats, ... On est obligés de rembourser les créanciers, mais pas les préteurs. Les intérêts sont ce que gagne les créanciers, les préteurs, et les dividants ce que gagnent les propriétaires des entreprises. Il faut aussi compter la fiscalité dans la répartition, l'entreprise verse aussi sa valeurs ajoutés aux services publiques : l'état à travers l'impôts, et les cotisations patronales, ce qui permet de payés la couvertures sociales aux salariés.
\paragraph{} Les banquiers / créanciers ne prêtent de l'argent que si les propriétaires y apportent un apport.
\paragraph{} On paye donc les salaires, les impôts, et les intérets et on reçoit la valeurs ajoutés. Le résultat donne des bénéfices, que l'on verse aux dividandes, et pour l'autofinancement.
~//
Se sont les actionnaires qui ont gagné les conflits de "celui qui veut gagner le plus".
\paragraph{} Entre 1949 et 2009, la part des salaires dans la valeurs ajoutés est très grandes durant les 30 glorieuses (termine en 1973). Les salaires correspondait quasiment à 75\% de la valeurs ajoutés. Il y a ensuite une rupture brutale, une remise à plat du conflits sociale entre le patronat les syndicats : les salaires ont chuté comparé à la valeurs ajouté : on passe de 75\% à 66\% . Cela s'explique par le fait que les salaires était indéxé au niveau de l'inflation avant les années 80 puis au niveau de l'entreprise après les années 80 : pouvoir d'achat en chute.
\section{Les cycles dans l'entreprises}
\paragraph{Cycle d'exploitation} L'entreprises / entrepreneurs prennent des risques. Etre entrepreneurs est risqué (surtout au moment de la création de l'entreprise). Ses couts interviennent avant ce qu'ils gagnent. On part donc avec une trésorie en générale négative. Les recettes devraient être supérieurs à nos couts et notre trésorerie est positive. Il faut pour cela au début, avoir des prets, des banques qui nous suivent.
\paragraph{cycle d'investimment} qui sert à plusieurs cycle d'exploitation. Les investiment peuvent être matériels (les locaux) ou immatériel (les brevets)
\paragraph{cycle de financement} Rythme perpétuels d'apport de capital des actionnaires notamment, et le remboursement des créanciers.
\paragraph{} Entreprises = ensemble de personnes à objectifs partagé (Plus facile de les partager dans les PME que dans les grandes entreprisent). Dimension sociale / manageariale très grandes. Division du travail, dans le pouvoir, dans les salariés. Dimensions de pouvoir : pouvoir informelle, avec une personne pas très haut dans l'entreprise mais très reconnu pour ses compétenses, et le pouvoir formel qui est celui qui est en haut de l'entreprise. Tout ses acteurs non pas forcément les mêmes intérets.

\section{Les fonctions par nature de l'entreprises}

\section{L'entreprise et son environnement} : Une entreprise est un macro environnementi (cadre, conjoncture économique, contexte sociologique, cadre international, evolutions techniques) et aussi un micro environnement (clients, fournisseurs, salariés, partenaire, concurrents) C'est un environnement où l'entreprise peut agir, contrôler la stratégie alors que la macro environnement est plus générale.
