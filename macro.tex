%tikz Macro

\usepackage{xargs}
\usepackage{calc}
\usetikzlibrary{calc}
\makeatletter


\newcommand{\getxy}[3]{%
	\path (#3);\pgfgetlastxy{#1}{#2} 
}

\newcommand{\extractangle}[3]{%
    \pgfmathanglebetweenpoints{\pgfpointanchor{#2}{center}}
                              {\pgfpointanchor{#3}{center}}
	\global\let#1\pgfmathresult
}

\newcommand{\extractLength}[3]{%
	\pgfpointdiff{\pgfpointanchor{#2}{center}}
                 {\pgfpointanchor{#3}{center}}
    \pgfmathparse{veclen(\pgf@x,\pgf@y)}
    \global\let#1\pgfmathresult % we need a global macro
}

\newcommandx{\repere}[6][5=0.5, 6=1]{
	\draw[gray, very thin, step=#5] (#1) grid (#2);
	\getxy{\xtpp}{\ytpp}{#1}
	\getxy{\xtpm}{\ytpm}{#2}
	\draw[->, black, thick] (0,{\ytpp/#6}) -- (0,{\ytpm/#6});
	\draw[->, black, thick] ({\xtpp/#6},0) -- ({\xtpm/#6},0);
	\draw[->, red,thick] (0,0) -- (#3,0);
	\draw[->, red,thick] (0,0) -- (0,#4);
}

%pour les arbres comme pour les dossiers. Utiliser \begin{tikzpicture}[dirtree]
\newcount\dirtree@lvl
\newcount\dirtree@plvl
\newcount\dirtree@clvl
\def\dirtree@growth{%
  \ifnum\tikznumberofcurrentchild=1\relax
  \global\advance\dirtree@plvl by 1
  \expandafter\xdef\csname dirtree@p@\the\dirtree@plvl\endcsname{\the\dirtree@lvl}
  \fi
  \global\advance\dirtree@lvl by 1\relax
  \dirtree@clvl=\dirtree@lvl
  \advance\dirtree@clvl by -\csname dirtree@p@\the\dirtree@plvl\endcsname
  \pgf@xa=1cm\relax
  \pgf@ya=-1cm\relax
  \pgf@ya=\dirtree@clvl\pgf@ya
  \pgftransformshift{\pgfqpoint{\the\pgf@xa}{\the\pgf@ya}}%
  \ifnum\tikznumberofcurrentchild=\tikznumberofchildren
  \global\advance\dirtree@plvl by -1
  \fi
}

\tikzset{
  dirtree/.style={
    growth function=\dirtree@growth,
    every node/.style={anchor=north},
    every child node/.style={anchor=west},
    edge from parent path={(\tikzparentnode\tikzparentanchor) |- (\tikzchildnode\tikzchildanchor)}
  }
}

\makeatother

\newcommand{\cfbox}[2]{%
    \colorlet{currentcolor}{.}%
    {\color{#1}%
    \fbox{\color{currentcolor}#2}}%
}
