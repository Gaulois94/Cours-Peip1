\section{POlynôme de degré 2 à  coefficient réels}

$P(x) = ax^2 + bx + c, (a, b, c) \in \mathbb{R}^3$

Discriminant $\Delta = b^2 - 4ac$

\begin{itemize}
	\item[$\Delta > 0$] il y a 2 racines réels : $\frac{-b \pm \sqrt{\Delta}}{2a}$
	\item[$\Delta = 0$] il existe une racine double : $\frac{-b}{2a}$
	\item[$\Delta < 0$] il n'y a pas de racines réels, mais il existe 2 racines réels conjugués $\frac{-b \pm i\sqrt{-\Delta}}{2a}$
\end{itemize}

\paragraph{Remarque}, il existe une seul formule : $\sqrt{-b \pm \delta}{2a}$ avec $\delta$ tel que $\delta^2 = \Delta$.

\paragraph{Exemple}

\[\begin{array}{rcl}
	x^2 + 1 &=& 0 \\
	\Delta &=& 0 - 4 = -4 < 0\\
	\text{Les racines sont} &:& \frac{-0 \pm i\sqrt{4}}{2} = \pm i\end{array}\]

\paragraph{Exemple} $x^2 -2x + 2$

\[\begin{array}{rcl}
	\Delta &=& 4 - 4\cdot 2 = -4 < 0 \\
	\text{2 racines complexe} &:& \frac{\sqrt{2 \pm i\sqrt{4}}}{2} = 1 \pm i\end{array}\]

\section{Fonctions complexes}

\[f : A \rightarrow \mathbb{C} \text{ avec } A \cup \mathbb{C}\]

$Re, Im, |z| \mathbb{C}  \rightarrow \mathbb{R}$

\[\left.\begin{array}{rcl}
	\mathbb{C} &\rightarrow & \mathbb{C} \\
	z &\to& \overline{z}
\end{array}\right) \text{Symétrie d'axe } O_x\]

\[
\begin{array}{rcl}
\exp : \mathbb{R} &\rightarrow& \mathbb{C} \\
\theta &\to& e^{i\theta} = \cos(\theta) + i \sin(\theta)\end{array}\]

\[
\begin{array}{rcl}
\exp : \mathbb{R} &\rightarrow& \mathbb{R} \\
x &\to& e^{x}\end{array}\]

\begin{wrapfigure}[6]{c}{0pt}
	\begin{tikzpicture}
		\draw[] (-2, 0) -- (4, 0);
		\draw[] (0, -2) -- (0, 2);
		\draw[] (0, 0) circle(1);
		\draw[thick, red] (0, 0) (30:1) node[right] {$e^{i\theta}$};
	\end{tikzpicture}
\end{wrapfigure}

On cherche à définir une fonction exponentielle $\exp : \mathbb{C} \rightarrow \mathbb{C}$

tel que \[\begin{array}{rcl}
\forall(z_1, z_2) &\in& \mathbb{C}^2 \\
e^{z_1 + z_2} &=& e^{z_1}\cdot e^{z_2}\end{array}\]

En particulier, on a pour $z = x+iy$

\[\begin{array}{rclr}
	e^z &=& e^x \cdot e^{iy} \\
		&=& re^{i\theta} \\
		e^z &=& e^{Re(z)} (\cos(Im(2)) + i \sin(Im(2))) & \text{Definition de l'exponentielle complexe} \\
		|e^z| &=& e^{Re(z)}
\end{array}\]

\paragraph{Remarque} comme pour les fonctions réelles, $f : A \rightarrow \mathbb{C}$

\begin{itemize}
	\item f est injective si pour tous $z_1 \in A, z_2 \in A, z_1 \neq z_2 \Rightarrow f(z_1) \neq f(z_2)$

	\item f est surjective si $\forall z \in \mathbb{Z}, \exists z' \in A, \text{ tq } f(z') = z$
	\item f est bijective si elle est injective et surjective.
\end{itemize}

\paragraph{Exemple} $\exp : \mathbb{C} \rightarrow \mathbb{C}$ n'est pas surjective : $|e^z| = e^{Re(z)} > 0$. 0 n'a donc pas d'antécédent par $\exp$

n'est pas injective.
\[\begin{array}{rclr}
z_1 = x + iy \neq z_2 &=& x+i(y+2\pi) \\
e^{z_2} = e^x \cdot e^{i(y+2\pi)} &= e^x \cdot e^{iy} \\
&&& = e^{z_2}\end{array}\]

\paragraph{Remarque}
On a bien:
\[\begin{array}{rl}
	\forall z_1 \in \mathbb{C},& \forall z_2 \in\mathbb{C} \\
e^{z_1 + z_2} =& e^{z_1}\cdot e^{z_2}
\end{array}\]

\[\forall z \in \mathbb{C}, \overline{e^z} = e^{\overline{z}}\]

\[\begin{array}{rcl}
	z &=& x+iz \\
\overline{e^{z}} &=& \overline{e^x \cdot e^{iy}} = \overline{e^x} \cdot \overline{e^{iy}} \\
&=& e^x \cdot e^{iy} \\
&=& e^{x-iy} = e^{\overline{z}}\end{array}\]

\section{Equation différentielles, linéaires d'ordre 1. homogènes et complexes}

\subsection{Cas particulier des fonctions à valeur complexes}

$f : I \rightarrow \mathbb{C}, I$ intervalle ouvert de $\mathbb{R}$.

On peut écrire pour tout $t \in I$

\[f(t) = \underbrace{Re(f(t))}_{f_1} + i\underbrace{Im(f(t))}_{f_2}\]

avec $f_1, f_2 : I \rightarrow \mathbb{R}$

\paragraph{Remarque}

\[\begin{array}{rcl}
	f_1 &=& \frac{1}{2}(f + \overline{f}) = Re o f \\
	f_2 &=& \frac{1}{2}(f - \overline{f}) = Im o f
\end{array}\]

\paragraph{Proposition} $f : I \rightarrow \mathbb{C}, t_0 \in I$

f est continue en $t_0$ si et seulement si $Re(f), Im(f) I \rightarrow \mathbb{R}$ sont continues en $t_0$

\paragraph{Proposition} $f : I \rightarrow \mathbb{C}, t_0 \in I$

f est dérivable en $t_0$ si et seulement si $Re(f), Im(f) I \rightarrow \mathbb{R}$ sont dérivable en $t_0$
ET $f'(t_0) = (Re(f))'(t_0) + i (Im(f))'(t_0)$

\paragraph{Proposition} $f : I \rightarrow \mathbb{C}, t_0 \in I$

f est intégrable en $I$ si et seulement si $Re(f), Im(f) I \rightarrow \mathbb{R}$ sont intégrables sur $I$
ET $\int f(t)dt = (\int Re(f(t))dt) + i (\int Im(f(t))dt)$

\paragraph{Remarque}

$z_n \rightarrow l$ avec $z_n \in \mathbb{C}, l \in \mathbb{C}$

\[\Rightarrow \left\{\begin{array}{rcl}
	Re(z_n) &\rightarrow& Re(l) \\
	Im(z_n) &\rightarrow& Im(l)\end{array}\right.\]

\paragraph{Exemple}

\[\begin{array}{rcl}
	\mathbb{R} &\rightarrow& \mathbb{C} \\
	t &\mapsto& e^{it} = \cos t + i \sin t
\end{array}\]

est continue sur $\mathbb{R}$ et de classe $C^\infty$

\[\begin{array}{rcl}
	f(t) &=& e^{it} = \cos t + i \sin t \\
	f'(t) &=& (\cos 't) + i(\sin 't) \\
		&=& -\sin t + i \cos t \\
		&=& i(\cos t + i \sin t) \\
		&=& i\cdot(e^{it}) \\
							\\
	\text{et } \int f(t)dt &=& (\int\cos t dt) + i (\int \sin t dt) \\
		&=& \sin t - i \cos t \\
		&=&  \frac{1}{i}(i \sin t + \cos t) \\
		&=& \frac{1}{i} e^{it} + (c_1 + i c_2)
\end{array}\]

\subsection{Equation différentielle d'ordre 1 homogène}

\paragraph{Théorème} Soit $a : I \rightarrow \mathbb{C}$

(I intervalle ouvert de $\mathbb{R}$) continue. Les solutions de l'équation différentielle \[(E) z' = az\]
Sont \[\left\{\begin{array}{rclr}
z : I & \rightarrow& \mathbb{R} \\
t &\mapsto & De^{A(t)} & D \in \mathbb{C}\end{array}\right.\]

et $A(t)=\int a(s)ds$ (une primitive de a).

\paragraph{Exemple} $E : z' = (2+8it)$

\[\begin{array}{rcl}
	a : \mathbb{R} & \rightarrow & \mathbb{C}\\
t &\mapsto & 2+8it\end{array}\]

Toutes les solutions de (E) sont les fonctions du type \[\begin{array}{rclr}
z : \mathbb{R} &\rightarrow& \mathbb{C}\\
z(t) &=& De^{A(t)} & D \in \mathbb{C}
\\
A(t) &=& \int a(s) ds = \int (2+8is)ds \\
&=& (\int 2ds) + i(\int 8s ds) \\
&=& 2t + i(4t^2) \\
z(t) &=& De^{(2t +4it^2)}
\end{array}\]

avec $D \in \mathbb{C}$
