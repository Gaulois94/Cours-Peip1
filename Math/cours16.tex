Pour g différent de 0, \[\frac{dy}{g(y)} = k(t)dt\]

Par "primitivisation" : \[\int \frac{dy}{g(y)} = \int h(t)dt\]

\paragraph{Exemple} \[\begin{array}{rcl}
		g : & y \mapsto	& y^2 \\
			& \mathbb{R} \to & \mathbb{R}^+ \\
\\
		k : &t \mapsto & 6t \\
			&\mathbb{R} \to \mathbb{R} \\
\\
		(E)	&y' = &6t y^2 
\end{array}\]

ou encore

\[\begin{array}{rclr}
	\int \frac{dy}{y^2}& = &\int 6t dt \\
	\text{Soit } -\frac{1}{y} &=& 3t^2 + c, & c \in \mathbb{R} \\
	\text{On obtient} y(t) &=& \frac{1}{-3t^2 - c} \\
							&=& \frac{1}{d - 3t^2}, & d \in \mathbb{R}
\end{array}\]

C'est à dire la fonction $y(t) = \frac{1}{d-3t^2}$ avec $d \in \mathbb{R}$ sont des solutions de (E), sur I (à préciser)

\paragraph{2ème étape} Utiliser les conditions initiales pour trouver d.

Conditions initiales : $y(0) = 2$ (on avait deux individus à l'instant $t=0$)	


En particulier, $y(0) = \frac{1}{d} = 2$ ~\\
soit $d = \frac{1}{2}$

D'où $y(t) = \frac{2}{1-6t^2}$ est une solution de (E) qui vérifie $g(0) = 2$

Problème : quel est l'intervalle I ?

La fonction y est définie sur : $]-\infty; -\frac{\sqrt{6}}{6}[ \cup ]-\frac{\sqrt{6}}{6}; \frac{\sqrt{6}}{6}[ \cup ]\frac{\sqrt{6}}{6}; +\infty[$

Ici, $y : t \mapsto \frac{2}{1-6t^2}$ est solution de (E) sur $]-\frac{\sqrt{6}}{6}; \frac{\sqrt{6}}{6}[$ et vérifie $y(0) = 2$

\paragraph{Remarque} On cherchait une fonction dérivable (pour $"y'"$ soit défini). Si on trouve y, en fait y est de classe $C^1$ (car $k(t)g(t)$ est continue)

\section{Équations différentielles linéaires, d'ordre 1}

$(E) :  \overbrace{y'}^{\text{ordre 1}} = a(t)\underbrace{y}_{\text{"linéaire"}} + \underbrace{b(t)}_{\text{Non homogène}}$ avec $a, b : I \to \mathbb{R}$ avec I une intervalle ouvert  et a, b des fonctions continue.

\paragraph{Résoudre (E)} c'est trouver toutes les fonctions $C^1$ qui vérifient (E) pour tout $t \in I$. Ici (par (E)), ceci est faisable.

\paragraph{Résolution}
\subsection{Équation homogène associée (on oublie "$b(t)$")}

\[(E_0) y' = a(t) y\]
Est une équations à variables séparables. Supposons que y ne s'annule pas.

On réécrit : $\int \frac{dy}{y} = \int a(t) dt$ ~\\
Soit $ln(|y|) = \int_{t_0}^t a(s)ds + c$ pour $c \in \mathbb{R}; t_0 \in I$

Via l'exponentielle, on obtient : \[|y(t)| = \exp(c+\int_{t_0}^t a(s)ds)\]

ou encore : \[\begin{array}{rclr}
|y(t)| &=& d \cdot e^{A(t)} & \text{avec primitive de a choisie}  \\
A(t) &=& \int_{t_0}^t a(s)ds & d \in \mathbb{R}^{+*}
\end{array}\]

\paragraph{Remarque} Si $y(t) = 0$, y solution de (E) alors $y'(t) = 0$

\paragraph{Remarque 2} Il y a trois cas mutuellement exclusif :
\begin{itemize}
	\item $y(t) = 0$ pour tout $t \in I; y(t) = 0\cdot e^{A(t)}$
	\item $y(t) > 0$ pour tout $t \in I; y(t) = d \cdot e^{A(t)}$ avec $d>0$
	\item $y(t) < 0$ pour tout $t \in I; y(t) = d \cdot e^{A(t)}$ avec $d<0$
\end{itemize}

$y(t)$ ne peux donc pas s'annuler sur I, car sinon il y(t) serait une constante pour tout $t \in I$.

\paragraph{} Les solutions de ($E_0$) sont les fonctions $y : I \rightarrow \mathbb{R}$ définie par $y(t) = de^{A(t)}$ avec $A(t) = \int_{t_0}^t a(s)ds$
\paragraph{Remarque} y est bien $C^1$

\subsection{Et le b ? : équations non homogène}

\paragraph{Théorème} Soit $a, b : I \rightarrow \mathbb{R}$ continue. Toutes les solutions de l'équation (E) \ul{sur I} sont les fonctions $C^1$ \[y(t) = e^{A(t)} (C + \int_\alpha^t e^{-A(t)} b(s)ds\]

où $\left\{\begin{array}{rcl}
	C &\in& \mathbb{R} \\
	\alpha &\in& I \\
	A(t) \text{ primitive de } a(t)\end{array}\right.$

	\paragraph{Démonstration} \begin{itemize}
		\item y est de classe $C^1$ sur I
\item reste à voir si y satisfait (E) ?\end{itemize}

	\[\begin{array}{rcl}
			y'(t) &=& {e^{A(t)}} '(C + \int_\alpha^t e^{-A(t)} b(s)ds + e^{A(t)} e^{-A(t)} b(t) \\
						   &=& a(t) y(t) + b(t) \\
	\text{car } y'(t) &=& a(t)e^{A(t)}(C + \int_\alpha^t e^{-A(t)}b(s)ds) + b(t)\end{array}\]

	Reste à voir si toutes les solutions sont de la forme ci dessus.

	On pose $z(t) = y(t) e^{-A(t)}$

	\[\begin{array}{rcl}
			z'(t) = (y\cdot e^{-A(t)})' &=& y'e^{-A(t)} + y(-a(t))e^{-A(t)} \\
				   &=& a(t)y + b(t)e^{-A(t)} - a(t)ye^{A(t)} \\
					  &=& b(t)e^{-A(t)} \end{array}\]

			Donc $z = \int b(s)e^{-A(s)} ds$, c'est à dire il existe une constante $C \in \mathbb{R}$ tel que \[z(t) = \int_\alpha^t b(s)e^{-A(s)}ds + C\]. D'où 
			\begin{center}
				\fbox{$y(t) = z(t)e^{A(t)} = e^{A(t)}(C+\int_\alpha^t e^{-A(s)}b(s)ds$)}
			\end{center}

			\paragraph{Théorème}
			Soit $a, b : I \rightarrow \mathbb{R}$ continue avec I un intervalle ouvert ~\\
			Il existe une unique solution de $y' = a(t) y + b(t)$ telle que pour $t_0$ et $y_0 \in \mathbb{R}, y(t_0) = y_0$

			\paragraph{Démonstration} C'est la solution avec $C_0$ qui vérifie : \[\begin{array}{rcl}
					y(t_0) &=& y_0 \\
					e^{A(t_0)}(C_0 + \int_\alpha^t e^{-A(s)bsds}) &=& y_0 \\
			\text{c'est à dire } C_0 &=& y_0e^{-A(t_0)} - \int_\alpha^t e^{-A(s)}b(s)ds\end{array}\]

Le tout étant des constantes, $C_0$ est une constante unique.

\paragraph{Remarque} Si on avait choisie $\alpha = t_0$, alors $C_0 = y_0e^{-A(t_0)}$ et la solution de (E) qui vérifie $y(t_0)=y_0$ est : \[y(t) = e^{A(t)}(y_0e^{-A(t_0)} + \int_{t_0}^t e^{-A(s)}b(s) ds)\]

\paragraph{Remarque} le système $\left\{\begin{array}{rcl} y' &=& a(t)y + b(t) \\
y(t_0) &=& y_0\end{array}\right.$ est nommé "Problème de Cauchy".

La théorème ci dessus est un cas particulière du théorème de Cauchy Lipschitz.
