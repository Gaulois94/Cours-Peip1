\paragraph{Remarque} $f^{-1}$ pourrait être la fonction $\frac{1}{f}$ (la fonction f est différent de 0), la fonction réciproque de f (avec f bijective).
~\\
$\begin{array}{rcl}
	\text{Pour } f : E & \rightarrow & \mathbb{R}, B \subset \mathbb{R} \\
	f^{-1}(B) & = & \{x \in E, f(x) \in \mathbb{R}\} \text{ Est toujours définie}
\end{array}$


\paragraph{Proposition} $ f : E \rightarrow F$ et $g:F \rightarrow G$
si f et g sont bijective, alors $gof$ l'est aussi et $(gof)^{-1} = g^{-1}of^{-1}$ $(gof : E \rightarrow G)$

\paragraph{Exemple} Trouver la fonction réciproque de $f : \mathbb{R} \rightarrow ]-7, +\infty[ $, $f(x) = e^{3x+2} - 7$
On écrit $y = e^{3x+2}-7$ et on détermine x en fonction des y.

\[\begin{array}{rclr}
	y + 7 & = & e^{3x+2} \\
		ln(y+7) & = & 3x + 2  \text{ y} > -7 & \text{ car fonction } exp > 0\\
	x &=& \frac{1}{3}(ln(y+7)-2)
\end{array}\]

d'où $f^{-1}(x) = \frac{1}{3}(ln(x+7)-2)$

\paragraph{Etablie} $f : E \rightarrow \mathbb{R}$ et $A \subset E$ ~\\
$f(A) = \{y \in \mathbb{R} \text{ tel que } x \in A, f(x) = y\}$
$f(A) = im(f_{|A})$

\chapter{Limites}

\section{Voisinage et adhérence}
\paragraph{Definition} si $x \in E$, on dit que E est un voisinage de x si E contient un intervalle ouvert qui contient x. Ceci est équivalent à E voisinage de x si il existe $\delta > 0$  tel que $]x - \delta; x + \delta[ \subset E$.

\paragraph{Définition} Soit $E \subset \mathbb{R}$. Un réel x est \ul{adherent} à E, si tout voisinage V de x intersecte E, c'est à dire $(V \cap E \neq \o)$

\paragraph{Exemple} 
\begin{itemize}
	\item si $x \in E$, x est adhérent à E, car pour tout voisinage V de x, $x \in V \cap E$
	\item E = ]0; 1], 0 est adhérent à E.
	\item $E = \{1 + \frac{1}{n}; n \in \mathbb{N}^*\} = \{2, \frac{3}{2}, \frac{4}{3}\}$
	1 est adhérent à E car $\lim_{n \to +\infty}1+\frac{1}{n} = 1$
\end{itemize}

\section{Limite finie en un point de $\mathbb{R}$}

\paragraph{Definition} $f : E \rightarrow \mathbb{R}; x_0$ un point adhérent de E. ~\\
On dit que f(x) \ul{tend vers} l en $x_0$ ou que f(x) \ul{admet la limite} l en $x_0$ si :
$\forall \epsilon > 0, \text{il existe } \delta > 0, |x - x_0| < \delta \rightarrow |f(x) - l| < \epsilon$
~\\
~\\ 
Ceci est équivalent à dire que $\forall \epsilon > 0, \text{ il existe } \delta > 0 \text{ tel que } \forall x \in [x_0-\delta, x_0+\delta], f(x) \in [l-\epsilon, l+\epsilon]$

Pour tout voisinage V de l il existe un voisinage de $x_0$ U tel que si x est dans U, alors f(x) est dans V.
\paragraph{Notation} $\lim_{x \to x_0} f(x) = l$ ou $f(x) \xrightarrow[x \to x_0]{} l$
\newpage
\paragraph{Exemple} $ f:\mathbb{R}^+ \rightarrow \mathbb{R}$ dont le graph est :

\begin{tikzpicture}
	\draw[->] (-3, 0) -- (3, 0);
	\draw[->] (0, -1) -- (0, 2);
	\draw[red] (-3, 1) -- (3, 1);
	\node at (0, 1) [fill=white] {1};
\end{tikzpicture}

\[ \lim_{x \to 0} f(x) = 1\] Soit $\epsilon > 0$, tout $\delta > 0$ convient.

\begin{wrapfigure}[0]{r}{0pt}
\begin{tikzpicture}
	\draw[->] (-3, 0) -- (3, 0);
	\draw[->] (0, -1) -- (0, 2);
	\draw[red] (0, 1) -- (3, 1);
	\draw[red] (-3, 0) -- (0, 0);
	\draw[fill=black] (0,0) circle (0.05);
	\draw[fill=white] (0,1) circle (0.05);
	\node at ([xshift=-0.4cm]0, 1) [fill=white] {1};
\end{tikzpicture}
\end{wrapfigure}

\[f(x) = 
\left\{
\begin{array}{r c l}
	0 & si & x \leq 0 \\
	1 & si & x > 0
\end{array}
\right.
\]

f n'admet pas de limite en 0.

\section{Restriction à un sous ensemble}
$f : E \rightarrow \mathbb{R}, A \subset \mathbb{R}, x_0 \text{ adhérent à A.}$
On dit que f(x) tend vers $l \in \mathbb{R}$ quand x tends vers $x_0$ dans A.

$\forall \epsilon > 0, \text{ il existe } \delta > 0,  \text{tel que } \forall x \in A,|x-x_0| < \delta, |f(x) - l| < \epsilon$

\paragraph{Exemple limite à gauche} de f en $x_0$ est $\lim_{\substack {x \to x_0 \\ x < x_0}} f(x)$ c'est à dire la limite de f(x) quand x tends vers $x_0$ dans $]-\infty, x_0[$
\paragraph{Exemple limite à droite} de f en $x_0$ est $\lim_{\substack{x \to x_0 \\ x > x_0}} f(x)$ c'est à dire la limite de f(x) quand x tends vers $x_0$ dans $]x_0, +\infty[$

\paragraph{Exemple}
La fonction f de l'exemple [x] admet une limite à droite en 0 : $\lim_{\substack{x \to 0 \\ x > 0}} f(x)$, pour $f(x) = 1$
La fonction f de l'exemple [x] admet une limite à gauche en 0 : $\lim_{\substack{x \to 0 \\ x < 0}} f(x)$, pour $f(x) = 0$

\paragraph{Remarque} On écrit aussi $\lim_{x \to x_0} f(x)$ par $\lim_{\substack{x \to x_0 \\ x > x_0}} f(x)$ et $\lim_{x \to 0} f(x)$ par $\lim_{\substack{x \to 0 \\ x > 0}} f(x)$

\section{Propriété}
\paragraph{Unicité} Si la limite existe, elle est unique.

démontration par l'absurde : $f : E \rightarrow \mathbb{R}, x_0$ adhérent à E. On suppose que la limite en $x_0$ existe mais qu'elle n'est pas unique.
Supposons que $\lim_{x \to x_0} f(x) = l_1 \text{et} \lim_{x \to x_0} f(x) = l_2$ avec $l_1 \neq l_2$

Comme $\lim_{x \to x_0} f(x) = l_1$, $\forall \epsilon_1 > 0$, il existe $\delta_1, \forall x \in E |x-x_0|< \delta_1, \text{alors} |f(x) - l_1| < \epsilon_1$ (*)
~\\
De plus $\lim_{x \to x_0} f(x) = l_2$, $\forall \epsilon_2 > 0$, il existe $\delta_2, \forall x \in E |x-x_0|< \delta_2, \text{alors} |f(x) - l_2| < \epsilon_2$ (**)
~\\

Choisissons $\epsilon < \frac{l_1+l_2}{2}$, on remarque $]l_1 - \epsilon, l_1 + \epsilon[ \cap ]l_2 - \epsilon, l_2 + \epsilon[ = \o$
~\\
	On trouve $\delta_1$ et $\delta_2$ tel que (*) et (**) soient vraies.
~\\
On appelle $\delta = min\{\delta_1, \delta_2\}, ]x_0-\delta; x_0+\delta[ \subset ]x_0 - \delta_1; x_0+\delta_1[ \cap ]x_0-\delta_2;x_0+\delta_2[$
~\\

Soit $x \in ]x_0-\delta; x_0+\delta[$ Par (*), $f(x) \in ]l_1-\epsilon; l_1+\epsilon[$
~\\
et par (**), $f(x) \in ]l_2-\epsilon; l_2+\epsilon[$ donc $f(x) \in ]l_1 - \epsilon, l_1 + \epsilon[ \cap ]l_2 - \epsilon, l_2 + \epsilon[ = \o $ Ceci est absurde ($f(x) \neq \o$)
~\\

\section{Théorème des gendarmes}
f, g, h 3 fonctions $E \rightarrow \mathbb{R}$, $x\in \mathbb{R}$ adhérent à E. \newline
(i) Si f, g, h admettent pour limites respectivs l, m, n en $x_0$ et si f(x) $\leq g(x) \leq h(x)$ pour tout x de E, alors $l \leq m \leq n$
~\\
(ii) Si $f(x) \leq g(x) \leq h(x)$ sur E et si f et h admettent une limite (identique) l en $x_0$, alors g admet une limite en $x_0$ et $\lim_{x \to x_0} g(x) = l$

\paragraph{Remarque} On remplace les inégalité de (i) par $\forall x \in E$, $f(x) < g(x) < h(x)$, on obtient aussi $l \leq m \leq n$

\paragraph{Exemple} $f(x) = |x|$ et $g(x) = 2\dot |x|$ Sur $E \subset R^+, f<g$ mais $\lim_{x \to 0}f(x) = \lim_{x \to 0} g(x)$

\paragraph{Exemple} \[\lim_{x \to 0} x .\sin(\frac{1}{x}) \text{ existe ?}\]
($\sin(\frac{1}{x})$ n'a pas de limite en 0)

Soit f, g, h $\mathbb{R}^* \rightarrow \mathbb{R}$, $f(x) = -|x|, g(x) = x \sin(\frac{1}{x}), h(x) = |x|$

On a bien $\forall x \in \mathbb{R}^*, f(x) \leq g(x) \leq h(x)$ car $\forall x \in \mathbb{R}, -1 \leq \sin(x) \leq 1$

Donc par le théorème des gendarmes, Comme $\lim_{x \to 0} f(x) = 0 \text{ et } \lim_{x \to 0} h(x) = 0$ g admet 0 comme limite quand x tends vers 0.
