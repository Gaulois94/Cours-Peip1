\paragraph{Fonction de référence}
\[\begin{array}{rccl}
	f: &\mathbb{R} &\rightarrow &\mathbb{R} \\
	& x & \rightarrow &x^n 
\end{array}\]
\[\lim_{x \to 0} f(x) = 0\]


\[\begin{array}{rccl}
	f: &\mathbb{R}^{+*} &\rightarrow &\mathbb{R} \\
	   & x & \rightarrow &x^{ln(n)} 
\end{array}\]
\[\lim_{x \to 0} f(x) = 0\]

\[\begin{array}{rccl}
	f: &\mathbb{R}^{+*} &\rightarrow &\mathbb{R} \\
	   & x & \rightarrow &x^{\alpha} \cdot ln(x)^\beta
\end{array}\]
\[\alpha >0, \beta>0, \lim_{x \to 0} f(x) = 0\]

\[\begin{array}{rccl}
	f: &\mathbb{R}^* &\rightarrow &\mathbb{R} \\
	   & x & \rightarrow &\frac{\sin{x}}{x}
\end{array}\]
\[\lim_{x \to 0} f(x) = 1\]

\paragraph{Methode}
\[\begin{array}{rccl}
	f:&\mathbb{R} &\rightarrow &\mathbb{R}\\
	  &x& \rightarrow & \frac{1}{x} - \frac{1}{x(x+1)}
\end{array}\]

$f(x) = \frac{(x+1)-1}{x(x+1)} = \frac{x}{x+1} = \frac{1}{x+1}$
donc $\lim_{x \to 0}f(x) = 1$

\[\begin{array}{rccl}
	f:&\mathbb{R}^+ &\rightarrow &\mathbb{R}\\
	  &x& \rightarrow & \frac{\sqrt{3+x} - \sqrt{3}}{2x}
\end{array}\]

\[\begin{array}{rccl}
	f(x) &=& \frac{\sqrt{3+x} - \sqrt{3}}{2x} . \frac{\sqrt{3+x} + \sqrt{3}}{\sqrt{3+x} + \sqrt{3}} \\
			   &=& \frac{\sqrt{3+x}^2 - \sqrt{3}^2}{2x(\sqrt{3+x} + \sqrt{3})} \\
			   &=& \frac{1}{2(\sqrt{3+x}+\sqrt{3})} \\
		\text{Donc } \lim_{x \to 0} f(x) &=& \frac{1}{4\sqrt{3}}
\end{array}\]

~\\
~\\

\paragraph{Comportement local}
\paragraph{Proposition} Si f(x) admet une limite $l \in \mathbb{R}$ quand x tends vers $x_0$, alors f est localement bornée.
c'est à dire il existe un voisinnage de x, V, tel que il existe $M \in \mathbb{R}, \forall x \in V, |f(x)| < M$
\paragraph{Remarque} Il existe un voisinnage de $x_0$ si et seulement si il existe un intervalle ouvert contenant $x_0$ si et seulement si il existe $\delta > 0, x \in ]x_0 - \delta, x_0 + \delta[$

\paragraph{Demonstration}
Par hypothèse, $f(x) \xrightarrow[x \to x_0]{} l$

c'est à dire $\forall \epsilon > 0, \text{ il existe } \delta > 0 \text{ tel que } |x-x_0| < \delta$   $|f(x)-l| < \epsilon$

Soit $\epsilon = 1$, On trouve $\delta$ tel que $\forall x \in ]x_0 - \delta, x_0 + \delta[ $

	$|f(x) - l| < 1$, c'est à dire $-1 < f(x) - l < 1$ Soit $|f(x)| < l+1$ 

\paragraph{Propriété}Si f(x) admet $l \neq 0$ comme limite quand x tends vers $x_0$, alors localement (autour de $x_0$), alors f est de signe constant
\paragraph{Démonstration} bornée en $x_0$ (meme style que la précédente), $\epsilon = \frac{l}{3}$

\paragraph{Exemple} \[\lim_{x \to 1} f(x) = 6 = f(1) \text{ avec } f=x^2+2x+3\]
\[\begin{array}{rcll}
	|f(1+h) - f(1)| &=& |(1+h)^2 + (1+h)*2 + 3 - 6| \\
	&=& |1+2h+h^2 + 2 + 2h -3| \\
	&=& |h(h+4)| \\
	&=& |h|*(h+4)  & \text{      si } |h| < 1 \\
	&\leq& 5|h|\\
	&& \lim_{h \to 0} 5|h| = 0
\end{array}\]

Par le théorème des gendarmes, \[\lim_{h \to 0} |f(1+h)-f(1)| =0\]

\paragraph{Remarque} $x = 1+h$ quand h tends vers 0 et x tends vers 1.

\section{Opération sur les limites}

$f, g : E \rightarrow \mathbb{R}; x_0 \text{ adherent à E}$
Supposons que $\lim_{x \to x_0} f(x) = l, \lim_{x \to 0} g(x) = m$ Alors 
\begin{itemize}
	\item $\lim_{x \to x_0} (f+g)(x) \text{ existe et vaut } l+m$
	\item $\lim_{x \to x_0} (f.g)(x) \text{ existe et vaut } l.m $
	\item si $m \neq 0$, alors $\lim_{x \to x_0} (f/g)(x) \text{ existe et vaut } \frac{l}{m}$
\end{itemize}

~\\
Composition $f:E \rightarrow F, g:F \rightarrow G$
~\\
$gof : E \rightarrow G$, $x_0$ adhérent à E.
~\\

Supposons que \begin{itemize}
	\item $\lim_{x \to x_0} f(x) = l \in \mathbb{R}$
	\item F est un voissinage de l.
	\item $\lim_{y \to l} g(y) = m$
\end{itemize}

Alors $\lim_{x \to x_0} gof(x)$ existe et vaut m.

\paragraph{Exemple} \[\begin{array}{rccl}
	g:& y&\rightarrow & e^y \\
		f:& x&\rightarrow & \sqrt{1+x}
\end{array}\]

\begin{itemize}
	\item gof est bien défini car le domaine de g est $\mathbb{R}$
	\item 0 est bien adhérent  au domaine de f (qui est $[-1, +\infty[$)
	\item $\lim_{x \to 0} f(x) = l$
	\item $\lim_{y \to 1} g(x) = e$
\end{itemize}

\section{Limites infinies, et limites en l'infinie}
\paragraph{Définition} $f:E\rightarrow \mathbb{R}, x_0$ adhérent à E

On dit que f(x) tend vers $+\infty$ (ou $-\infty$) quand x tend vers $x_0$ si $\forall A > 0, \text{ il existe } \delta > 0 \text{ tel que } |x - x_0| < \delta$, alors $f(x) > A$ (ou $f(x) < -A$ pour f(x) tend vers $-\infty$).

\paragraph{Exemple} \[\lim_{\substack{x \to 0 \\ x > 0}} \frac{1}{x} = +\infty\]
~\\

\[\lim_{\substack{x \to 0 \\ x < 0}} \frac{1}{x} = -\infty\]

\paragraph{Définition} $f:E \rightarrow \mathbb{R}$ tel qu'il existe $A >0$ tel que $]A; +\infty[ \subset E$
	On dit que f(x) tend vers $l \in E$ quand x tend vers $+\infty$ 
	~\\
	c'est à dire $\forall \epsilon > 0, \text{ il existe } A >0, x > A, \text{ alors } |f(x) - l| < \epsilon$

\paragraph{Définition} $f:E \rightarrow \mathbb{R}$ tel qu'il existe $A <0$ tel que $]-\infty, A[ \subset E$
	On dit que f(x) tend vers $l \in E$ quand x tend vers $-\infty$ 
	c'est à dire $\forall \epsilon > 0, \text{ il existe } A <0, x < A, \text{ alors } |f(x) - l| < \epsilon$

	\paragraph{Remarque} $\lim_{x \to -\infty} f(x) = +\infty$ veut dire 
	$\forall A >0, \text{ il existe } B > 0, x < -B$ tel que $f(x) > A$

\paragraph{Exemple}

\[\begin{array}{rccl}
	f:& \mathbb{R}^* &\rightarrow & \mathbb{R} \\
	x&\mapsto& \frac{1}{x}
\end{array}\]

$\lim_{\substack{x \to 0 \\ x > 0}} f(x) = +\infty$
\paragraph{Démonstration} Soit $A>0$. On cherche $\delta$ tel que si $0<x$, $0<\delta$ alors $f(x) = \frac{1}{x} > A$

Choisir $\delta = \frac{1}{A}$ suffit, en effet $0<x<\frac{1}{A}$ alors $\frac{1}{x} > \frac{1}{A}$.

\paragraph{Exemple}
$g(x) = 1+e^{-x}$ Montrons que $\lim_{x \to +\infty} g(x) = 1$ et $\lim_{x \to -\infty}g(x) = +\infty$

\paragraph{Exemple}
\[\begin{array}{rccl}
f:& ]-\frac{\pi}{2}; \frac{\pi}{2}[ &\rightarrow & \mathbb{R} \\
  &	x&\rightarrow& tan(x)
\end{array}\]

\[\lim_{\substack{x \to -\frac{\pi}{2} \\  x > - \frac{\pi}{2}}} tan(x) = -\infty\]

\section{Opération sur les limites}
\begin{itemize}
	\item Limites finies ($l\in \mathbb{R}$) en l'infini sont exactement les memes opérations.
	\item Limites infinies ($l = \pm \infty$) Attention aux cas inderminé :
		~\\
		$+\infty - \infty, \frac{\pm \infty}{\pm\infty}, 0*(\pm\infty)$
\end{itemize}

\paragraph{Exemple} $\frac{+\infty}{+\infty} = ?$

\[\begin{array}{rcl}
	f:x & \mapsto & x \\
	g:x &\mapsto & x^2
\end{array}\]

\[\lim_{x \to +\infty} \frac{f(x)}{g(x)} = 0\]

\[\begin{array}{rcl}
	f_2:x &\mapsto & x^3 \\
	g_2x&\mapsto& x^2
\end{array}\]

\[\lim_{x \to +\infty} \frac{f_2(x)}{g_2(x)} = +\infty\]

\[\begin{array}{rcl}
	f_3:x &\mapsto & 3x \\
	x&\mapsto& x
\end{array}\]

\[\lim_{x \to +\infty} \frac{f_3(x)}{g_3(x)} = 3\]

Plus généralement, P,Q deux polynomes, que vaut $\lim_{x\to +\infty} \frac{P(x)}{Q(x)}$ ? Elle est égale au rapport des thermes du plus haut degrés.
Exemple : \[\lim_{x \to +\infty} \frac{3x^2 -2x + 4x^5 +2}{x^4+3} = \lim_{x \to +\infty} \frac{4x^5}{x^4} = +\infty\]

\paragraph{Exemple}
\[\lim_{x \to 0} x*\sin(\frac{1}{x}) = 0\]

car $\forall x \neq 0$, $1 \leq \sin(\frac{1}{x}) \leq 1$
~\\
donc $ 0 \leq |x*\sin(\frac{1}{x})| \leq |x|$ avec |x| tend vers 0 pour x tend vers 0.
