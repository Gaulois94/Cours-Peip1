\subsection{b' et le b ? Variation de la constante}

\paragraph{Rappel} Solution de $y'=a(t)y$ est $C\exp({A(t)})$ avec $A = \int^t a(s)ds$ et $C \in \mathbb{R}$

On cherche une solution de $y'=a(t)y+b(t)$. On fait "varier" la constante. C'est à dire on considère $C(t)e^{A(t)}=y(t)$

Alors si $y$ est bien solution de (E), \[\begin{array}{rclr}
y'&=&a(t)y + b(t) \\
C'(t)e^{A(t)} + C(t)(a(t)\cdot e^{A(t)}) &=& a(t)C(t)e^{A(t)} + b(t) \\
\text{On obtient } C'(t)e^{A(t)} &=& b(t) \\
\text{Soit } C'(t)&=& b(t)e^{-A(t)} \\
\text{Donc } C(t) &=& \int^t_\alpha b(s)e^{-A(s)} ds + D & D \in \mathbb{R} \end{array}\]

\subsection{Équations non-résolues}

\[u(t)y' = v(t)y+w(t) \text{ } (F)\] Avec $u, v, w : I \rightarrow \mathbb{R}$ continues, (I un intervalle ouvert).

On travaille sur les intervalles où $u$ ne s'annule pas. ~\\
Sur $J \subset I$ J un intervalle ouvert où u ne s'annule pas, on réécrit (F) comme $y' = \underbrace{\frac{v(t)}{u(t)}}_{a(t)}y + \underbrace{\frac{w(t)}{u(t)}}_{b(t)} $

On regarde si on peut "raccorder" les solutions de manière $C^1$

\paragraph{Remarque} On perd l'unicité des solutions (même en imposant des conditions initiales)

\paragraph{Exemple} \[(\sin t)y' - (\cos t)y = e^t\cdot\sin^2t \text{ } (E)\]

$(E)$ n'est pas résolue !

On se restreint aux intervalles où $\sin t \neq 0$. Par exemple $\underbrace{]-\pi, 0[}_{J^-}$ et $\underbrace{]0, \pi[}_{J^+}$

Sur $J^-$ ou $J^+$, (E) s'écrit \[y' = \frac{\cos t}{\sin t}y + e^t \sin t (F)\]

Les solutions de l'équation différent homogène associée \[y' = \frac{\cos t}{\sin t} y \text{ } (E_0)\]

Les solutions sont \[\begin{array}{rclr}
y(t) &=& C \exp(\int^t_\alpha \underbrace{\frac{\cos s}{\sin s}}_{\frac{\sin't}{\sin t}}ds) &\text{sur} J^+ (\alpha \in J^+) \\
y(t) &=& C \exp(\int^t_\alpha \frac{\sin's}{\sin s}ds) \\ 
y(t) &=& C \exp([\ln(|\sin t|)]^t_\alpha) \\ 
y(t) &=& C \exp([\ln(|\sin t|)]^t_{\frac{\pi}{2}}) & \text{On choisi } \alpha = \frac{\pi}{2}
\end{array}\]

Soit $\begin{array}{rclr}y(t) &=& C\exp(\ln(|\sin t|)) \\
&=& C|\sin t| & C \in \mathbb{R}\\
&=& C\cdot \sin t & \text{sur } J^+\end{array}$

Sur $J^-$, on choisit $\alpha = -\frac{\pi}{2}$ et on obtient \[\begin{array}{rclr}
y(t) &=& C\cdot |\sin t|&, C \in \mathbb{R} \\
&=& D \cdot \sin t&, D \in \mathbb{R}\end{array}\]

Résolution de l'équations non homogène.

Sur $J^+$

On cherche juste \[\begin{array}{rcl}
&&\int e^{-A(s)}b(s)ds \\
&=& \int \frac{1}{e^{ln({\sin s})}} \cdot e^s \sin s ds \\
&=& \int \frac{1}{\sin s} \cdot e^s \cdot \frac{\sin^2 s}{\sin s} ds \\
&=& \int e^s ds \\
&=& e^t + C\end{array}\]

D'où la solution particulière étant :
\[\begin{array}{rcl}
y &=& e^{A(t)}\int e^{-A(s)}b(s) ds \\
&=& e^{A(t)}\cdot(e^t + C) \\
&=& \sin t (e^t + C)\end{array}\]

D'où la solution générale de (F) sur $J^+$ est :

\begin{center}
	\fbox{$y(t) = \sin t (C+e^t), C\in \mathbb{R}$}
\end{center}

De même, sur $J^-$ la solution générale de (F) est : $y(t) = \sin t(D+e^t), D\in \mathbb{R}$

Existe-t-il des fonctions continues vérifiant (F) sur $I = ]-\pi, \pi[$

si une telle fonction f existe alors f et en particulier solution sur $J^-$ et $J^+$ donc \[\left\{\begin{array}{rcl}
	\sin t(C + e^t) & \text{si } t \in J^- \\
	\sin t(D + e^t) & \text{si } t \in J^+\end{array}\right.\] 

			On veut prolonger en 0 :\[\lim_{\substack{t \to 0 \\ t > 0}} f(t) = 0 = \lim_{\substack{t \to 0 \\ t < 0}} f(t)\]

De plus
\[\begin{array}{rclr}
	f'(t)  &=& \sin t(C+e^t) + \sin t(e^t) & \text{Sur } J^-\\
		&=& \cos t(C +e^t) + \sin t \cdot e^t \\
\text{Donc } \lim_{\substack{t \to 0 \\ t < 0}} f'(t) &=& C+1 = \lim_{\substack{t \to 0 \\ t > 0}} f'(t) \end{array}\]

Par le théorème de prolongement $C^1$, f est $C^1$ si et seulement si $C=D$

Donc les solutions de $(E)$ sur $I = ]-\pi, \pi[$ sont les fonctions \[f(t) = \sin t(C+e^t), C\in \mathbb{R}\]

\paragraph{Remarque} Il y a une infinité de solutions de (E) sur $]-\pi, \pi[$ tel que $y(0)=0$. Cependant, aucune ne satisfait $y(0)=3$

Finalement, si $t_0 \in J^-$ ou $J^+$ et si $y_0 \in \mathbb{R}$. Il y a une unique solution qui satisfait $y(t_0)=y_0$

\chapter{Les nombres complexes}
	\section{Définitions, représentations}


\begin{tikzpicture}
	\draw[red] (-2, 0) -- (2, 0) node [midway, below right] {$\mathbb{R}$};
	\draw[blue] (0, -2) -- (0, 2) node [midway, above left] {$I$};
	\draw[green] (0, 0) -- (45:1.44) coordinate(A) node [midway, above] {$r$};
	\getxy{\x}{\y}{A}
	\draw[green] (0, 0) -- (-45:1.44) node [midway, above] {$r$};
	\draw[fill=black] (A) circle (0.05) node [right] {$z$};
	\draw[dashed] (A) --(0,\y) node [left] {$y$}; 
	\draw[dashed] (A) --(\y, 0) node [below right] {$x$}; 

	\draw[fill=black] (\x, -\y) circle (0.05) node [right] {$\overline{z}$};
	\draw[dashed] (\x, -\y) --(0,-\y) node [left] {$-y$}; 

	\draw[blue] (0.5, 0) arc (0:45:0.5) node [right] {$\theta$};
	\draw[blue] (0.5, 0) arc (0:-45:0.5) node [right] {$-\theta$};
\end{tikzpicture}

On note i un nombre tel que $i^2 = -1$

\begin{tabular} {r|c|l}
{} & Euclidien & Polaire \\
\hline
$z \in \mathbb{C}$ & $x+y, x, y \in \mathbb{R}$ & $re^{i\theta}$ \\
$Re(z)$  &$x$&   $r \cos \theta : r \geq 0, \theta \in \mathbb{R}$ \\
$Im(z)$	& $y$ & $r \sin \theta$ \\
$|z|$ & $\sqrt{x^2 + y^2}$ & r \\
$arg(z)$&$\arccos(\frac{x}{\sqrt{x^2 + y^2}})=\arcsin(\frac{y}{\sqrt{x^2 + y^2}}) $ & $\theta$ \\
$\overline(z)$ : conjugué & $x-iy$ & $r\cdot e^{-i\theta}$ \\
nombre réels & $x \in \mathbb{R}$ & $r$ \\
imaginaire purs & $iy, y \in \mathbb{R}$ & $re^{i(\frac{\pi}{2} + k\pi)}, k \in \mathbb{Z}$
\end{tabular}

\paragraph{Propriétés} \[z = x+iy = re^{i\theta}, z' = x'+iy' = r'e^{i\theta '}\]

\begin{itemize}
	\item $z = z'$
		\begin{itemize}
			\item $x = x'$ et $y=y'$
			\item $r=r'$ et $\theta = \theta' + 2k\pi, k \in \mathbb{Z}$
		\end{itemize}
	Donc $z=0 \left\{ \begin{array}{rcl}
					x &=& 0 \text{ et } y = 0\\
					r &=& 0\end{array}\right.$
	
	\item[addition] \[z+z' = (x+x') + i(y+y')\]
	\item[multiplication] \[\begin{array}{rcl}
		z\cdot z' &=& (xx' - yy') + i(xy'+x'y) \\
&=& rr'e^{i\theta \cdot \theta'}\end{array}\]

	\item[multiplication par $a \in \mathbb{R}$]
		\[\begin{array}{rclr}
			az &=& (ax) + i(ay) \\
&=& (|a|r)e^{i\theta} & \text{si } a > 0 \\
&=& (|a|r)e^{i(\theta + \pi)} & \text{si } a < 0 \end{array}\]

	\item[conjugués et opération]
		\[\begin{array}{rcl}
				\overline(z+z') &=& \overline(z) + \overline(z') , \overline(-z) = -\overline(z) \\
				\overline(z\cdot z') &=& \overline(z) \cdot \overline(z') \\
				z+\overline(z) &=& 2Re(z) \\
				z - \overline(z) &=& 2 Im(z) \\
		z\overline(z) &=& Re(z)^2 + Im(z)^2 = |z|^2\end{array}\]
\end{itemize}
