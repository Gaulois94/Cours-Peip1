\documentclass{beamer}
 
\usepackage[frenchb]{babel}
\usepackage[T1]{fontenc}
\usepackage[utf8]{inputenc}
\usepackage{tikz}
\usepackage{calc}
\usepackage{mathrsfs}
\usepackage{amsmath}
\usepackage{amssymb}
\usepackage{wrapfig}
\usepackage{tikz}
\usetikzlibrary{intersections}
\usetikzlibrary{patterns}
\usetikzlibrary{positioning,shapes.geometric,shapes.symbols,shapes.misc,shapes.multipart}
\usetikzlibrary{trees}
\usepackage{listings}
\newcommand{\includecode}[2]{\lstinputlisting[caption=#2, escapechar=, style=custom#1]{#2}}
\lstdefinestyle{customc++}{
	language=C++,
	basicstyle=	\ttfamily\scriptsize,
	keywordstyle=\bfseries\color{green!40!black},
	commentstyle=\itshape\color{purple!40!black},
	identifierstyle=\color{blue},
	stringstyle=\color{orange},
	breaklines=true,
	numbers=left,
	tabsize=2,  
	stepnumber=1
}

\lstdefinestyle{customc}{
	language=C,
	basicstyle=	\ttfamily\scriptsize,
	keywordstyle=\bfseries\color{green!40!black},
	commentstyle=\itshape\color{purple!40!black},
	identifierstyle=\color{blue},
	stringstyle=\color{orange},
	numbers=left,
	tabsize=2,  
	breaklines=true,
	stepnumber=1
}

\lstdefinestyle{custompython}{
	language=Python,
	basicstyle=	\ttfamily\scriptsize,
	keywordstyle=\bfseries\color{green!40!black},
	commentstyle=\itshape\color{purple!40!black},
	identifierstyle=\color{blue},
	stringstyle=\color{orange},
	numbers=left,
	tabsize=2,  
	breaklines=true,
	stepnumber=1
}

\lstdefinestyle{customjava}{
	language=Java,
	basicstyle=	\ttfamily\scriptsize,
	keywordstyle=\bfseries\color{green!40!black},
	commentstyle=\itshape\color{purple!40!black},
	identifierstyle=\color{blue},
	stringstyle=\color{orange},
	numbers=left,
	tabsize=2,  
	breaklines=true,
	stepnumber=1
}


%tikz Macro

\usepackage{xargs}
\usepackage{calc}
\usetikzlibrary{calc}
\makeatletter

\newcommand{\getxy}[3]{%
	\path (#3);\pgfgetlastxy{#1}{#2} 
}

\newcommand{\extractangle}[3]{%
    \pgfmathanglebetweenpoints{\pgfpointanchor{#2}{center}}
                              {\pgfpointanchor{#3}{center}}
	\global\let#1\pgfmathresult
}

\newcommand{\extractLength}[3]{%
	\pgfpointdiff{\pgfpointanchor{#2}{center}}
                 {\pgfpointanchor{#3}{center}}
    \pgfmathparse{veclen(\pgf@x,\pgf@y)}
    \global\let#1\pgfmathresult % we need a global macro
}

\newcommandx{\repere}[6][5=0.5, 6=1]{
	\draw[gray, very thin, step=#5] (#1) grid (#2);
	\getxy{\xtpp}{\ytpp}{#1}
	\getxy{\xtpm}{\ytpm}{#2}
	\draw[->, black, thick] (0,{\ytpp/#6}) -- (0,{\ytpm/#6});
	\draw[->, black, thick] ({\xtpp/#6},0) -- ({\xtpm/#6},0);
	\draw[->, red,thick] (0,0) -- (#3,0);
	\draw[->, red,thick] (0,0) -- (0,#4);
}

\makeatother

\include{diagrammeShapev2}
 
\title{Bar robotisé partie robot CLEAN}
\author{SERENO Mickael}
\institute{Lycée Jean Macé}
\date{\today}

\logo{\includegraphics[height=12mm]{logo.jpg}}

\usetheme{Warsaw}
\hypersetup{pdfpagemode=FullScreen}

\setbeamertemplate{background canvas}[vertical shading][top=white, bottom=blue!40]

\AtBeginSection[]
{
  \begin{frame}
  \transdissolve
  \frametitle{Sommaire}
  \tableofcontents[currentsection, hideothersubsections, pausesubsections]
  \end{frame} 
}
 
\begin{document}
    \begin{frame}
		\titlepage
    \end{frame}

	\section{Présentation Générale}
	\subsection{Les buts du robot}
		\begin{frame}
			\frametitle{Récapitulatif des buts du robot}
			\transdissolve
			\transduration{4}
			\begin{itemize}
				\pause
				\item Prendre les déchets des consommateurs.
				\pause
				\item Se repérer sur la piste pour déposer ces déchets.
				\pause
				\item Faire attention aux autres robots.
				\pause
				\item Nettoyer à sec la piste.
			\end{itemize}
		\end{frame}

	\subsection{Le montage du robot}
		\begin{frame}
			\transdissolve
			\transduration{20}
			\begin{center}
				Le robot est composé de 8 ports pour actionneurs et 8 pour les capteurs. Nous n'utiliserons que 2 servomoteurs et 3 capteurs. On l'a aussi équipé d'un balai pour nettoyer à sec la piste (comme préciser dans le cahier des charges).
				\pause
				\includegraphics[height=36mm]{arriere.jpg}
			\end{center}
		\end{frame}
	\section{Choix technique}
	\subsection{Vitesse modulable}
		\begin{frame}
			\transdissolve
			\begin{tikzpicture}[scale=0.2]
				\coordinate (A) at (-1,-15);
				\coordinate (B) at (46,21);
				\repere{A}{B}{5}{10}[1][0.2]

				\foreach \x in {0, 50, ..., 450} 
				{
					\draw[red] (\x mm, -15) -- (\x mm, 21);
					\node at (\x mm, 0) [below] {$\x$};
				}
				\foreach \y in {-150, -100, ..., 200} 
				{
					\draw[red] (-1, \y mm) -- (46, \y mm);
					\node at (0, \y mm) [left] {$\y$};
				}
				\node at (45cm, 0) [right, yshift=-0.5cm, xshift=0.5cm] {mm};
				\node at (0, 20cm) [above, yshift=0.25cm, xshift=-0.5cm] {mm/s};

				\draw [domain=-2:47, blue, thick] plot (\x, {15/(45-20)*\x - 20*15/(45-20)}) coordinate (A);
				\pause
				\node at (A) [blue, xshift=-5cm, yshift=-1cm, above] {$f(x) = \cfrac{15}{45-20} * x - \cfrac{20*15}{45-20}$};
			\end{tikzpicture}
		\end{frame}
	\subsection{Prendre la Pince}
		\begin{frame}
			\transdissolve
			\transduration{4}
			Pour prendre la pince on :
			\begin{itemize}
				\pause
				\item{Tourne jusqu'à détecter la limite de la piste}
				\pause
				\item{On tourne jusqu'à détecter l'objet}
				\pause
				\item{Si on détecte les bandes blanches, c'est qu'il n'y a pas de déchets et donc on se remet dans l'axe}
				\pause
				\item{Sinon on avance, on baisse la pince, on avance encore et on ferme la pince}
				\pause
				\item{On identifie la nature de l'objet avec le capteur haut puis on monte la pince}
				\pause
				\item{Je retourne à ma position initiale et dans mon orientation initiale}
			\end{itemize}
		\end{frame}
	\section{Amélioration}
	\subsection{Gestion des utilisateurs}
		\begin{frame}
			\transdissolve
			\frametitle{Un tableau dynamique}
				\begin{tabular}{|c|}
					\hline
					Position Consommateur 1 \\
					\hline
					Position Consommateur 2 \\
					\hline
					Position Consommateur 3\\
					\hline
				\end{tabular}
				\begin{wrapfigure}[16]{r}{0pt}
				\begin{tikzpicture}
					\pause
					\node (begin) at (0,-1) [draw] {Consommateur 4};
					\pause
					\draw [->, red] (begin.south) to[out=-70,in=-10] (-8,-5mm);
					\node (espace) at (-9, -5mm) [minimum width=2] {};
				\end{tikzpicture}
			\end{wrapfigure}
		\end{frame}

		\begin{frame}
			\transdissolve
			\frametitle{Un tableau statique}
			\begin{minipage}[c]{\linewidth}
				\begin{tabular}{|c|}
					\hline
					Position Consommateur 1 \\
					\hline
					-1\\
					\hline
					-1\\
					\hline
					-1\\
					\hline
				\end{tabular}
			\begin{tikzpicture}
				\pause
				\node (begin) at (0,0) {Consommateur 3};
				\pause
				\draw [->, red] (begin.south) to[out=-40,in=-10] (-3,-16mm);
			\end{tikzpicture}
		\end{minipage}
		\end{frame}

		\subsection{De le Piste}
		\begin{frame}
			\transdissolve
			\begin{tikzpicture}
				\node (rectangle) at (0.5,0.5) [draw, rectangle] {Clean};
				
				\draw[thick, red] (2.2,1) -- (2.2,0) coordinate (L);
				\draw[thick] (2.2cm,0) -- (2.2cm, -1cm);
				\draw[thick, red] (2.2cm,-1cm) |- (3cm,-1.2cm) coordinate (A);
				\draw[thick] (3cm, -1.2cm) -- (4cm, -1.2cm);
				\draw[thick, red] (4cm,-1.2cm) -- (5cm,-1.2cm) coordinate (B);
				\draw[thick] (5cm, -1.2cm) -- (6cm, -1.2cm);
				\draw[thick, red] (6cm,-1.2cm) -- (7cm,-1.2cm) coordinate (C);
				\draw[thick] (7cm, -1.2cm) -- (8cm, -1.2cm);
				\draw[thick, red] (8cm,-1.2cm) -| (8.8cm,-1cm) coordinate (D);
				\draw[thick] (8.8cm, -1cm) -- (8.8cm, 0);
				\draw[thick, red] (8.8cm,0cm) -- (8.8cm,1cm) coordinate (E);


				\draw[thick] (8.8cm, 1cm) -- (8.8cm, 2);
				\draw[thick, red] (8.8cm,2cm) |- (8cm,2.2cm) coordinate (F);
				\draw[thick] (8cm, 2.2cm) -- (7cm, 2.2cm);
				\draw[thick, red] (7cm,2.2cm) -- (6cm,2.2cm) coordinate (G);
				\draw[thick] (6cm, 2.2cm) -- (5cm, 2.2cm);
				\draw[thick, red] (5cm,2.2cm) -- (4cm,2.2cm) coordinate (H);
				\draw[thick] (4cm, 2.2cm) -- (3cm, 2.2);
				\draw[thick, red] (3cm,2.2cm) -| (2.2cm,2cm) coordinate (I);

				\draw[thick] (2.2,2) -- (2.2,1);

				\draw[thick] (4.5cm, -1.2cm) -- (4.5cm, 0);
				\draw[thick, red] (4.5cm,0cm) -- (4.5cm,1cm) coordinate (J);
				\draw[thick] (6.5cm, 1cm) -- (6.5cm, 2.2);
				\draw[thick] (6.5cm, -1.2cm) -- (6.5cm, 0);
				\draw[thick, red] (6.5cm,0cm) -- (6.5cm,1cm) coordinate (K);
				\draw[thick] (4.5cm, 1cm) -- (4.5cm, 2.2);

				\pause

				\node[align=center, xshift=-5mm, below, black] at (A) {0};
				\node[align=center, xshift=-5mm, below, black] at (B) {1};
				\node[align=center, xshift=-5mm, below, black] at (C) {2};
				\node[align=center, yshift=-1mm, right, black] at (D) {3};
				\node[align=center, xshift=3mm, yshift=-5mm, right, black] at (E) {7};
				\node[align=center, xshift=5mm, above, black] at (F) {11};
				\node[align=center, xshift=5mm, above, black] at (G) {10};
				\node[align=center, xshift=5mm, above, black] at (H) {9};
				\node[align=center, xshift=5mm, yshift=2mm, above, black] at (I) {8};

				\node[align=center, xshift=-3mm, yshift=-5mm, left, black] at (J){5};
				\node[align=center, xshift=3mm, yshift=-5mm, right, black] at (K) {6};
				\node[yshift=5mm, xshift=-1mm, align=center, left, black] at (L) {4};

				\pause

				\draw[thick, blue, ->] (9.2, 0) -- (9.2,1);
				\draw[thick, blue, <-] (4.3, 0) -- (4.3,1);
				\draw[thick, blue, <-] (2, 0) -- (2,1);
				\draw[thick, blue, ->] (6.7, 0) -- (6.7,1);
			\end{tikzpicture}
		\end{frame}

		\section{Algorigramme de niveau 1}
		\begin{frame}
			\transdissolve
			\begin{center}
			\begin{tikzpicture}[node distance=0.7cm and 2cm]
				\node(declaration) at (0,0) [loop, align=center, scale=0.7] {Type inPince=Nothing \\
				LineSensor sensorVal \\
				entier rotationValue = 0 \\
				entier robotPosition=-1 \\ 
				entier goTo=4 \\
				entier vitesse=0 \\ 
				Couleur sensorColor=Black \\ 
				tableau entier costumer[8] \\ 
				entier nbCostumer};

				\node [below of=declaration, yshift=-0.3cm, start-end] (begin) {Debut} edge (declaration);
				\node [below of=begin, operation] (init) {Initialisation du robot} edge (begin);

				\pause
				\node [below of=init, operation] (setMotor) {On se déplace a vitesse maximum} edge(init);
				\node [below of=setMotor, yshift=-0.3cm, decision] (whiteLine) {Ligne Blanche ?} edge (setMotor);
				\node (no) at (whiteLine.east) [connection, text width=0cm, xshift=3pt] {};

				\draw[->] (no.east) -| ([xshift=1cm, yshift=0.5cm]no.east) |- ([yshift=0.1cm]whiteLine.north);

				\node [below of=whiteLine, operation, yshift=-3mm] (rotation) {faire une rotation de $-90^\circ$} edge (whiteLine);

				\pause

				\node [below of=rotation, yshift=-0.3cm, decision] (true) {Vrai ?} edge (rotation);
				\node(connection1) at ([xshift=2pt]true.east) [connection] {};
				\node [right of=connection1, start-end, text width=2cm, xshift=2cm] (end) {Fin} edge (connection1);
				\node [below of=true, macro, yshift=-2mm] (update) {
					\nodepart{two}update()} edge(true);
				\node [below of=update, operation] (continue) {Appliquer les valeurs de rotations et la vitesse du robot} edge(update);
				\draw[->] (continue.south) --+ (0,-0.1) -| ([xshift=-1cm]true.west) |- ([yshift=0.1cm]true.north);
			\end{tikzpicture}
		\end{center}
		\end{frame}

		\begin{frame}
			\transdissolve
		\begin{center}
			\begin{tikzpicture}[node distance=0.7cm and 2cm]
				\node(begin) at (0,0) [start-end] {Update};
				\node [below of=begin, operation] (MAJCo) {Mise à jours de la position et de la couleur} edge (begin);
				\node [below of=MAJCo, operation] (MAJR) {Mise à jours de la rotation du robot} edge (MAJCo);
				\node [below of=MAJR, operation] (majv) {Mise à jours de la vitesse} edge (MAJR);

				\pause

				\node [below of=majv, decision, yshift=-0.3cm] (bouton) {Bouton actionné ?} edge (majv);
				\node(connection1) at ([yshift=-2pt]bouton.south) [connection] {};
				\node [right of=bouton, operation, xshift=5cm] (enre) {Enregistre la position du robot} edge (bouton);
				\node [below of=bouton, decision, yshift=-1.3cm] (selon) {Café dans Pince ?} edge (connection1);
				\draw [->] (enre.south) |- ([yshift=0.1cm]selon.north);
				\node(connection2) at ([yshift=-2pt]selon.south) [connection] {};
				\node [right of=selon, decision, xshift=5cm, xscale=0.8] (plusRapide) {Position Robot = Zone 2 ?} edge (selon);
				\node(connection3) at ([yshift=-2pt]plusRapide.south) [connection] {};
				\node[right of=plusRapide, operation, xshift=4cm] (tourner) {Ce mettre dans l'axe vers la zone 6} edge (plusRapide);

				\node [below of=connection3, decision, xscale=0.8] (versZone) {Position Robot = Zone 9 ?} edge (connection3);
				\node[right of=versZone, operation, xshift=4cm] (tourner5) {Ce mettre dans l'axe vers la zone 5} edge (versZone);
				\node(connection4) at ([yshift=-2pt]versZone.south) [connection] {};

				\node [below of=connection4, decision, xscale=0.8] (poseCafe) {Position Robot = Zone 6 ?} edge (connection4);
				\node(connection5) at ([yshift=-2pt]poseCafe.south) [connection] {};
				\node [right of=poseCafe, operation, xshift=4cm] (pose) {On pose le cafe} edge (poseCafe);

				\getxy{\x}{\y}{connection2.south}
				\coordinate (A) at ([yshift=-0.1cm]connection5.south);
				\getxy{\xa}{\ya}{A}
				\draw [->] (tourner.east) -- ([xshift=0.25cm]tourner.east) |- ([yshift=-0.1cm]connection5.south) |- (\x,\ya);
				\node [below of=selon, connection, yshift=-8.3cm] {1} edge (connection2);
				\draw [] (connection5.south) -- ([yshift=-0.1cm]connection5.south);
				\draw [] (tourner5.east) -- ([xshift=0.25cm]tourner5.east);
				\draw [] (pose.east) -- ([xshift=0.25cm]pose.east);
			\end{tikzpicture}
			\end{center}
	\end{frame}

	\begin{frame}
		\transdissolve
		\begin{center}
		\begin{tikzpicture}[node distance = 0.5cm and 9cm]
			\node(begin) at (0,0) [connection] {1};
			\node[below of=begin, decision, yshift=-2mm, scale=0.9] (pinceSoda) {Soda dans Pince ?} edge (begin);
			\node(connection) at ([yshift=-2pt]pinceSoda.south) [connection] {};

			\node [right of=pinceSoda, decision, xshift=5cm, xscale=0.8] (zone2) {Position Robot = Zone 2 ?} edge (pinceSoda);
			\node(connection2) at ([yshift=-2pt]zone2.south) [connection] {};
			\node[right of=zone2, operation, xshift=5cm] (tourner) {Ce mettre dans l'axe vers la zone 6} edge (zone2);

			\node [below of=connection2, decision, xscale=0.8, yshift=-1mm] (versZone) {Position Robot = Zone 9 ?} edge (connection2);
			\node[right of=versZone, operation, xshift=5cm] (tourner5) {Ce mettre dans l'axe vers la zone 5} edge (versZone);
			\node(connection3) at ([yshift=-2pt]versZone.south) [connection] {};

			\node [below of=connection3, decision, xscale=0.8, yshift=-1mm] (poseSoda) {Position Robot = Zone 5 ?} edge (connection3);
			\node(connection4) at ([yshift=-2pt]poseSoda.south) [connection] {};
			\node [right of=poseSoda, operation, xshift=5cm] (pose) {On pose le soda} edge (poseSoda);

			\getxy{\x}{\y}{connection.south}
			\coordinate (A) at ([yshift=-0.05cm]connection4.south);
			\getxy{\xa}{\ya}{A}
			\draw [->] (tourner.east) -- ([xshift=0.25cm]tourner.east) |- ([yshift=-0.05cm]connection4.south) |- (\x,\ya);
			\draw [] (connection4.south) -- ([yshift=-0.05cm]connection4.south);
			\draw [] (tourner5.east) -- ([xshift=0.25cm]tourner5.east);
			\draw [] (pose.east) -- ([xshift=0.25cm]pose.east);

			\pause

			\coordinate (B) at (\x,\ya);
			\node (pinceN) at ([yshift=-0.6cm]B) [decision] {Rien dans la Pince ?} edge (connection);
			\node (connectionEnd) at ([xshift=2pt]pinceN.east) [connection] {};
			\node [right of=connectionEnd, xshift=5cm, start-end] (end) {Fin Fonction} edge (connectionEnd);

			\pause
			\node [below of=pinceN, yshift=-0.5cm, decision, scale=0.75] (decisionRapide) {Croisement et chemin plus rapide vers le premier consommateur ?} edge (pinceN);
			\node [right of=decisionRapide, xshift=5cm, operation] (cheminRapide) {On tourne et ce place dans l'axe du croisement} edge (decisionRapide);
			\node(connection5) at ([yshift=-2pt]decisionRapide.south) [connection] {};

			\draw[->] (cheminRapide.south) |- ([yshift=-0.1cm]connection5.south);
			\node[below of=connection5, yshift=-0.3cm, decision, scale=0.9] (consommateur) {Zone robot = zone du premier consommateur ? } edge (connection5);
			\node[right of=consommateur, operation, xshift=6cm] (debarasser) {On vérifie s'il y a un objet et on le prend} edge (consommateur);
			\node(connection6) at ([yshift=-2pt]consommateur.south) [connection] {};

			\node [below of=connection6, start-end, yshift=-1mm] (end2) {Fin Fonction} edge (connection6);
			\draw[->] (debarasser.south) |- ([yshift=0.1cm]end2.north);
		\end{tikzpicture}
		\end{center}

	\end{frame}
\end{document}
