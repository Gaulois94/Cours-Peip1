\chapter{Raisonner, Rédigé}

\paragraph{} Ce chapitre d'ouverture a quatre objectifs :
\begin{itemize}
	\item Vous apprendre ou vous rappeler les règles de base de la "grammaire" mathématique.
	\item Vous apprendre ou vous rappeler quelques rudiments de théorie des ensembles.
	\item Vous apprendre ou vous rappeler les raisonnements de base utilisés en mathématiques.
	\item Vous convaincre qu'il est essentiel de savoir rédiger : un peu pour "faire joli", mais surtout pour bien penser.
\end{itemize}

\section{Connecteurs Logiques}
\begin{itemize}
	\item Nous appellerons proposition toute phrase p au sujet de laquelle on peut poser la question :"p est-elle vraie ?". La plupart des phrases grammaticalement correctes sont des propositions, mais par exemple, "Dis-le moi! ", "Bonjours" ou "Comment vas tu?" n'en sont pas : "Est-il vrai que bonjours ?" n'a aucun sens.
	\item La valeur de vérité d'une proposition est soit le vrai (V), soit le faux (F). Deux proposition qui ont la même valeur de vérité sont dites équivalentes.

	\begin{minipage}[c]{\linewidth}
	\item
	\begin{wrapfigure}{r}{0pt}
		\begin{tabular}{|c|c|c|}
			\hline
			p&q&p et q \\
			\hline
			V&V&V\\
			\hline
			V&F&F\\
			\hline
			F&V&F\\
			\hline
			F&F&F\\
			\hline
		\end{tabular}
	\end{wrapfigure}
		Un connecteur logique est dit \emph{vérifonctionnel} si la valeur de vérité d'une proposition construite à l'aide de ce connecteur dépend sulement de la valeur de vérité des propositions utilisées dans la constriuction. Pour savoir, par exemple, si la proposition "p et q" est vraie, on n'a pas besoin de savoir exactement ce que cachent p et q. Leur significations. Seules leurs valeurs de vérité respectives importent : si les deux sont vrai, "p et q" est vrai, si l'une est fausse, "p et q" est fausse. \newline En mathématique, les connecteurs logiques sont tous vérifonctionnels. L'intérêt de tels connecteurs réside dans la facilité avec laquelle on peur les définir : au moyen d'un simple tableau appelé table de vérité.
	\end{minipage}
\end{itemize}
\subsection{\bf{NÉGATIONS} non, \bf{CONJONCTION} et, \bf{DISJONCTION ou}}

\fcolorbox{green!50!black}{white}{\begin{minipage}[c]{\linewidth}
					\begin{wraptable}{r}{0pt}
					\begin{tabular}{|>{\columncolor{yellow}}c|c|}
							\hline
							p & non p\\
							\hline
							V & F\\
							\hline
							F & V\\	
							\hline
						\end{tabular}
						\begin{tabular}{|c|c|}
							\hline
							p & non p\\
							\hline
							V & F\\
							\hline
							F & V\\	
							\hline
						\end{tabular}
					\end{wraptable}

		Définitions (Négations, conjonction, disjonction)

		La proposition "non p" est vrai si p est fausse.
			A l'inverse, ta gueule.
			Sinon la ferme
	\end{minipage}
	}
